\documentclass[ number=45
			   ,series=eotms
			   ,output=printondemand %  proof|printondemand   
			   %,blackandwhite
			   %,draft=yes
			  ]{langsci}                          
                               
\usepackage{layout}
\usepackage{lipsum} 
%\usepackage{tree-dvips}                                                          


%\title{Danish in \newlineCover Head-Driven \newlineCover\newlineSpine Phrase Structure \newlineCover  Grammar  }                        
\title{Ɓasaá in \newlineCover Head-driven \newlineCover\newlineSpine Phrase Structure Grammar}
%\title{Le language des fulɓe dans le cadre de la grammaire des syntagmes guidés par les têtes}

%\author{Stefan Müller, \newlineCover Pollet Samvelian, \newlineCover Olivier Bonami}
\author{Stefan Müller, \newlineCover Bjarne Ørsnes}
\BackTitle{Danish in Head-Driven Phrase Structure Grammar}
\BackBody{This book describes a fragment of the Danish language that is formalized in the framework of
Head-Driven Phrase Structure Grammar. Each chapter consist of a large empirical part that uses
corpus data for the discussion of phenomena whenever this is appropriate. The description is
theory-driven but the formulations of the insights are as theory-neutral as possible.

After a brief introduction into the basic assumptions that are made in HPSG, we provide a
topological model of the Danish clause. This topological model is used to explain the basic clause
structure of Danish (SVO) and to set up the HPSG analysis of these rather basic facts. After this
rather basic part detailed descriptions and analyses of the following phenomena are provided: object
and negation shift, copula constructions, including specificational structures and the relation to
left dilocation and question tag formation, extraposition, passive, raising passives, in which
objects of embedded verbs are raised to subject, preposed negation, extraction (V2), the insertion
of positional expletives in extraction contexts, and \emph{Do}-Support.

The grammar that is described in this book is computerprocessable and hence internally
consistent. It was developed in the CoreGram project, which develops grammars of various languages
and tries to find crosslinguistic generalizations. This is also reflected in the book, which
contains sections that compare the analyses for Danish with those for German and English.

\bigskip

Stefan Müller is professor for German and General Linguistics at the Freie Universität
Berlin. Bjarne Ørsnes is Associate Professor at the Copenhagen Business School. 
}                                            
 
\begin{document}               
        
                                    
                           
\maketitle                

\tableofcontents      
        
\part{Part Title: Ɓasaá, fulɓe, move-α, and the {\sc head} Feature}	               
\chapter{Chapter Title: Ɓasaá, fulɓe, move-α, and the {\sc head} Feature}
\section{Section Title: The details Ɓasaá, fulɓe, move-α, and the {\sc head} Feature}
\subsection{Subsection Title: more specific points on Ɓasaá, fulɓe, move-α, and the {\sc head} Feature}
\subsubsection{Subsubsection Title: Ɓasaá, fulɓe, move-α, and the {\sc head} Feature}       
\subsubsubsection{Subsubsubsection Title: Ɓasaá, fulɓe, move-α, and the {\sc head} Feature}       
                                                   

Hey, this is another text that is written by me, not be \LaTeX{}. It is much nicer, although the,
English may be broken and the interpuction incorrect.\footnote{
  The point of writing this text is that I can insert this footnote here so that Ulrike can see how
  a footnote looks like. The lipsum package does not insert footnotes.
}$^,$\footnote{
  And if they did not die, the wrote another footnote.
}$^,$\footnote{
  The point of writing this text is that I can insert this footnote here so that Ulrike can see how
  a footnote looks like. The lipsum package does not insert footnotes.
}


\lipsum 
\lipsum[3-10]  

 
%% \begin{tabular}{ccc}
%% \multicolumn{2}{c}{\node{0}{S}} \\[2ex]
%% \node{1}{NP} & \multicolumn{2}{c}{\node{2}{VP}} \\[2ex]
%% \node{11}{N} & \node{21}{AP*} & \node{22}{V$\downarrow$} \\[2ex]
%% \node{111}{\it John}
%% \end{tabular}
%% \nodeconnect{0}{1} \nodeconnect{0}{2} \nodeconnect{1}{11} \nodeconnect{11}{111}
%% \nodeconnect{2}{21} \nodeconnect{2}{22}  

\newpage

\layout
                              
\end{document}
      
