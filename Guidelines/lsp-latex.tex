%% -*- coding:utf-8 -*-
\chapter{\LaTeX}

\section{Installation of the \texttt{langsci} Class}

The \latex class for typesetting Language Science Press books was developed by Timm Lichte with
help be Berthold Crysmann and me. It can be downloded from the GitHUB repository at: \url{https://github.com/langsci/latex}

Place all files and subdirectories from this repository into your local working directory (for
advanced installations see Section~\ref{sec-advanced-latex-installation}).

\section{Using the \texttt{langsci} Class}

Once you installed the classes in your system, you may look at the file \texttt{test.tex} to see how
a book can be typeset. The code of this book is available in the directory \texttt{Guidelines}. Once
you set up your \latex files you can compile them by calling 
\begin{verbatim}
xelatex yourfilename.tex
\end{verbatim}


\section{Workflow}

\subsection{Compiling the Document}

\subsection{Makefiles}

\subsection{Using Includes}


\section{Document Structure}


\section{Packages specific for Linguistics}

\subsection{Glossed Examples}


\subsection{\texttt{jambox}}


\subsection{AVMs}

  \begin{avm}
    {\it word\/} $\rightarrow$
    \[morphs & $\@{e_1}\bigcirc\cdots\bigcirc\@{e_n}$\\
    morsyn & \@0 $(\@{m_1}\uplus\cdots\uplus\@{m_n})$\\
    rules & \< \[morphs & \@{e_1}\\mud & \@{m_1}\\ morsyn & \@0\],\ldots,
    \[morphs & \@{e_n}\\mud & \@{m_n}\\ morsyn & \@0\] \>
    \]
  \end{avm}

\subsection{Trees}

\subsection{OT Tableaux}


\begin{tabular}
       {|lc|c|c|c|}\hline   
      & \textbf{Input}  & Cnstrnt 1  &  Cnstrnt 2& Cnstrnt 3\\ \hline\hline
      & candidate 1     & *!         &           &          \\ \hline
      & candidate 2     &            &  *        &          \\ \hline
\hand & candidate 3     &            &           &  *       \\ \hline
\end{tabular}

\begin{fitverb}
\begin{verbatim}
\begin{tabular}
       {|lc|c|c|c|}\hline   
      & \textbf{Input}  & Cnstrnt 1  &  Cnstrnt 2& Cnstrnt 3\\ \hline\hline
      & candidate 1     & *!         &           &          \\ \hline
      & candidate 2     &            &  *        &          \\ \hline
\hand & candidate 3     &            &           &  *       \\ \hline
\end{tabular}
\end{verbatim}
\end{fitverb}

\verb+\hand+ ist wie folgt definiert:

\begin{verbatim}
\usepackage{pifont}
\newcommand{\hand}{\ding{43}}
\end{verbatim}
 


\begin{tabular*}{0.95\textwidth}
    {@{\extracolsep{\fill}}|rl||c|c|c|}\hline   
      & \textbf{Input} & Constraint 1 & Constraint 2 & Constraint 3 \\ \hline\hline
      & candidate 1    & *!           &              &              \\ \hline
      & candidate 2    &              &  *           &              \\ \hline
\hand & candidate 3    &              &              &  *           \\ \hline
\end{tabular*}

\begin{fitverb}
\begin{verbatim}
\begin{tabular*}{0.95\textwidth}
    {@{\extracolsep{\fill}}|rl||c|c|c|}\hline   
      & \textbf{Input} & Constraint 1 & Constraint 2 & Constraint 3 \\ \hline\hline
      & candidate 1    & *!           &              &              \\ \hline
      & candidate 2    &              &  *           &              \\ \hline
\hand & candidate 3    &              &              &  *           \\ \hline
\end{tabular*}
\end{verbatim}
\end{fitverb}

\begin{tabular}[t]{r|c|c|c|}
\cline{2-4}
      & /qi/  & qi    & qi         \\
\LCC 
      &       &       & \lightgray \\ \cline{2-4}
\hand & [qi]  &       & *          \\ \cline{2-4}
      & [*qi] & *!    &            \\ \cline{2-4}
\ECC
\end{tabular}


{\scriptsize
\begin{verbatim}
\usepackage{pstricks,colortab}

\begin{tabular}[t]{r|c|c|c|}
\cline{2-4}
      & /qi/  & qi    & qi         \\
\LCC 
      &       &       & \lightgray \\ \cline{2-4}
\hand & [qi]  &       & *          \\ \cline{2-4}
      & [*qi] & *!    &            \\ \cline{2-4}
\ECC
\end{tabular}
\end{verbatim}
}


\begin{tabular}{|l||c|c|} \hline
          &VO          &OV         \\ \hline\hline
\LCC
          &            &\lightgray \\ \hline
prefixing &Tagalog     &Ma'a       \\ \hline
\ECC
\LCC
           &\lightgray &            \\ \hline
suffixing  &Kwakwala   &Japanese    \\ \hline
\ECC
\end{tabular}

{\scriptsize
\begin{verbatim}
\begin{tabular}{|l||c|c|} \hline
          &VO          &OV         \\ \hline\hline
\LCC
          &            &\lightgray \\ \hline
prefixing &Tagalog     &Ma'a       \\ \hline
\ECC
\LCC
           &\lightgray &            \\ \hline
suffixing  &Kwakwala   &Japanese    \\ \hline
\ECC
\end{tabular}


\end{verbatim}
}



\subsection{Font Issues and Right to Left Scripts}


  
\section{Bells and Whistles}

\subsection{\texttt{varioref}}

\subsection{\texttt{german} for Hyphenation}

\subsection{\texttt{xspace} and Abbreviations}

\section{Software}

\begin{itemize}
\item BibDesk
\item JabRef
\end{itemize}


\section{Advanced Installation}
\label{sec-advanced-latex-installation}

If you typeset many books for \lsp, put the fonts that are contained in the directory \texttt{fonts}
into your local font directory that is used by \xelatex. Put the logos from the directory \texttt{logos} into the
search path for images. 


%      <!-- Local IspellDict: en_US-w_accents -->
