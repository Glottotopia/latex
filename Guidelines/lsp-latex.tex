%% -*- coding:utf-8 -*-
\chapter{\LaTeX}

\section{Installation of the \texttt{langsci} Class}

The \latex class for typesetting Language Science Press books was developed by Timm Lichte with
help be Berthold Crysmann and me. It can be downloded from the GitHUB repository at: \url{https://github.com/langsci/latex}

Place all files and subdirectories from this repository into your local working directory (for
advanced installations see Section~\ref{sec-advanced-latex-installation}).

\section{Using the \texttt{langsci} Class}

Once you installed the classes in your system, you may look at the file \texttt{test.tex} to see how
a book can be typeset. The code of this book is available in the directory \texttt{Guidelines}. Once
you set up your \latex files you can compile them by calling 
\begin{verbatim}
xelatex yourfilename.tex
\end{verbatim}


\section{Workflow}

\subsection{Compiling the Document}

\subsection{Makefiles}

\subsection{Using Includes}


\section{Document Structure}


\section{Packages specific for Linguistics}

\subsection{Glossed Examples}


\subsection{\texttt{jambox}}


The package \texttt{jambox} can be used to provide information about the language of an example or
about a certain other aspect to be highlighted.
\settowidth\jamwidth{VSO}
\eal
\ex[]{
\gll Ingrid kiel-et il-mazzit-a.\\
     Ingrid eat-3fsg def-black.pudding-fsg\\ \jambox{(SVO)}
\glt `Ingread ate black pudding.'
}
\ex[]{
Kielet il-mazzita Ingrid. \jambox{(VOS)}
}
\ex[*]{
Kielet Ingrid il-mazzita. \jambox{(VSO)}
}
\ex[]{\label{ex-sov}
Ingrid il-mazzita kielet. \jambox{(SOV)}
}
\ex[]{\label{ex-osv}
Il-mazzita Ingrid kielet. \jambox{(OSV)}
}
\ex[]{
Il-mazzita kielet Ingrid. \jambox{(OVS)}
}
\zl

The call of \verb+\jambox+ has to follow the linebreak after the gloss:
\begin{verbatim}
\ex[]{
\gll Ingrid kiel-et il-mazzit-a.\\
     Ingrid eat-3fsg def-black.pudding-fsg\\ \jambox{(SVO)}
\glt `Ingread ate black pudding.'
}
\end{verbatim}
The distance from the right margin can be specified by passing the largest object to be placed in a
jambox to \verb+\settowidth+:

\eal
\settowidth\jamwidth{(German)}
\ex The man reads the book.    \jambox{(English)}
\ex Manden læser bogen.        \jambox{(Danish)}
\ex Der Mann liest das Buch.   \jambox{(German)}
\zl

\begin{verbatim}
\eal
\settowidth\jamwidth{(German)}
\ex The man reads the book.    \jambox{(English)}
\ex Manden læser bogen.        \jambox{(Danish)}
\ex Der Mann liest das Buch.   \jambox{(German)}
\zl
\end{verbatim}

\subsection{AVMs}

  \begin{avm}
    {\it word\/} $\rightarrow$
    \[morphs & $\@{e_1}\bigcirc\cdots\bigcirc\@{e_n}$\\
    morsyn & \@0 $(\@{m_1}\uplus\cdots\uplus\@{m_n})$\\
    rules & \< \[morphs & \@{e_1}\\mud & \@{m_1}\\ morsyn & \@0\],\ldots,
    \[morphs & \@{e_n}\\mud & \@{m_n}\\ morsyn & \@0\] \>
    \]
  \end{avm}

\subsection{Trees}

\subsection{OT Tableaux}


\begin{tabular}
       {|lc|c|c|c|}\hline   
      & \textbf{Input}  & Cnstrnt 1  &  Cnstrnt 2& Cnstrnt 3\\ \hline\hline
      & candidate 1     & *!         &           &          \\ \hline
      & candidate 2     &            &  *        &          \\ \hline
\hand & candidate 3     &            &           &  *       \\ \hline
\end{tabular}

\begin{fitverb}
\begin{tabular}
       {|lc|c|c|c|}\hline   
      & \textbf{Input}  & Cnstrnt 1  &  Cnstrnt 2& Cnstrnt 3\\ \hline\hline
      & candidate 1     & *!         &           &          \\ \hline
      & candidate 2     &            &  *        &          \\ \hline
\hand & candidate 3     &            &           &  *       \\ \hline
\end{tabular}
\end{fitverb}

\verb+\hand+ ist wie folgt definiert:

\begin{verbatim}
\usepackage{pifont}
\newcommand{\hand}{\ding{43}}
\end{verbatim}
 


\begin{tabular*}{0.95\textwidth}
    {@{\extracolsep{\fill}}|rl||c|c|c|}\hline   
      & \textbf{Input} & Constraint 1 & Constraint 2 & Constraint 3 \\ \hline\hline
      & candidate 1    & *!           &              &              \\ \hline
      & candidate 2    &              &  *           &              \\ \hline
\hand & candidate 3    &              &              &  *           \\ \hline
\end{tabular*}

\begin{fitverb}
\begin{tabular*}{0.95\textwidth}
    {@{\extracolsep{\fill}}|rl||c|c|c|}\hline   
      & \textbf{Input} & Constraint 1 & Constraint 2 & Constraint 3 \\ \hline\hline
      & candidate 1    & *!           &              &              \\ \hline
      & candidate 2    &              &  *           &              \\ \hline
\hand & candidate 3    &              &              &  *           \\ \hline
\end{tabular*}
\end{fitverb}

\begin{tabular}[t]{r|c|c|c|}
\cline{2-4}
      & /qi/  & qi    & qi         \\
\LCC 
      &       &       & \lightgray \\ \cline{2-4}
\hand & [qi]  &       & *          \\ \cline{2-4}
      & [*qi] & *!    &            \\ \cline{2-4}
\ECC
\end{tabular}


\begin{verbatim}
\usepackage{pstricks,colortab}

\begin{tabular}[t]{r|c|c|c|}
\cline{2-4}
      & /qi/  & qi    & qi         \\
\LCC 
      &       &       & \lightgray \\ \cline{2-4}
\hand & [qi]  &       & *          \\ \cline{2-4}
      & [*qi] & *!    &            \\ \cline{2-4}
\ECC
\end{tabular}
\end{verbatim}


\begin{tabular}{|l||c|c|} \hline
          &VO          &OV         \\ \hline\hline
\LCC
          &            &\lightgray \\ \hline
prefixing &Tagalog     &Ma'a       \\ \hline
\ECC
\LCC
           &\lightgray &            \\ \hline
suffixing  &Kwakwala   &Japanese    \\ \hline
\ECC
\end{tabular}

\begin{verbatim}
\begin{tabular}{|l||c|c|} \hline
          &VO          &OV         \\ \hline\hline
\LCC
          &            &\lightgray \\ \hline
prefixing &Tagalog     &Ma'a       \\ \hline
\ECC
\LCC
           &\lightgray &            \\ \hline
suffixing  &Kwakwala   &Japanese    \\ \hline
\ECC
\end{tabular}
\end{verbatim}




\subsection{Font Issues and Right to Left Scripts}

Since we are using \xelatex, all fonts that are installed in the cannonical font directories can be
used. We are using the font \texttt{Linux Libertine}, which is unicode-based and contains a lot of
the characters linguists want to use.

\subsubsection{Chinese}


\ea
\glll 狗       叫     了。\\
      gou3     jiao4   le\\
      dog      bark    ASP/CRS\\
\glt `The dog is barking.'/`The dogs are barking.'
\z

\begin{verbatim}
\usepackage[indentfirst=false]{xeCJK}
\setCJKmainfont{SimSun}

\ea
\glll 狗       叫     了。\\
      gou3     jiao4   le\\
      dog      bark    ASP/CRS\\
\glt `The dog is barking.'/`The dogs are barking.'
\z
\end{verbatim}


\subsubsection{Arabic Script}



\ea
\PRL{او مرد را دوست نخواهد داشت.}\\
 \gll   U mard rā dust naxāhad dāšt.\\
   He/she man {\sc dom} friend {\sc neg}.want have\\
\glt `He/she will not love the man.'
\z

%\begin{rtlverbatim}
%\usepackage{fontspec}
\begin{verbatim}
\newfontfamily\Parsifont[Script=Arabic]{XB Niloofar}
\usepackage{bidi}
\newcommand{\PRL}[1]{\RL{\Parsifont #1}}

\ea
\PRL{او مرد را دوست نخواهد داشت.}\\
\gll U      mard rā       dust   naxāhad        dāšt.\\
     He/she man {\sc dom} friend {\sc neg}.want have\\
\glt `He/she will not love the man.'
\z
\end{verbatim}
%\end{rtlverbatim}

\subsubsection{IPA Symbols}

  
\section{Bells and Whistles}

\subsection{\texttt{varioref}}

\subsection{\texttt{german} for Hyphenation}

If you write things like \verb+head-driven+ or very long pathes like
{\sc snysem$|$""loc$|$""cat$|$""head$|$""mod$|$""loc}, \LaTeX{} does not do hyphenation
(in the part following the dash).

\verb+german.sty+ provides additional markup that allows for proper hyphenation:
\begin{verbatim}
head"=driven

{\sc snysem$|$""loc$|$""cat$|$""head$|$""mod$|$""loc}
\end{verbatim}
With this markup even long pathes like {\sc snysem$|$loc$|$cat$|$""head$|$""mod$|$""loc$|$""cat$|$""head}
are typeset properly. Alternatively you my write
\begin{verbatim}
{\sc snysem$|$\-loc$|$\-cat$|$\-head$|$\-mod}
\end{verbatim}
which introduces a dash at the place of the linebreak:
{\sc snysem$|$\-loc$|$\-cat$|$\-head$|$\-mod$|$\-loc$|$\-cat$|$\-head}.

If you use \verb+german.sty+ for a book whose primary language is not German, do not forget to
specify the language you are using. For example, if your book is in US English you have to specify
the following:
\begin{verbatim}
\selectlanguage{USenglish}
\end{verbatim}
Otherwise the section name for references comes out in German.

\subsection{Resizing Large Objects}

Trees or AVMs often are too big to fit onto one page. The \texttt{langsci} comes with commands for
shrinking large objects. You may pass your complex object as an argument to \texttt{\oneline} and
this will scale the object to \verb+\linewidth+ (the remaining space on the current line). There is
a more clever version of this command: \verb+\centerfit+. This command checks whether there is
enough space for an object and if this is the case it centers it in the line. If the object is
larger than the \verb+\linewidth+, it is resized to fit the line. This is very handy for typesetting
figures. You may copy and paste figures to other documents with a different text width without any
adaptions.


%% \begin{figure}[htb]
%% \centerfit{%
%% \begin{tikzpicture}
%% \tikzset{level 1+/.style={level distance=3\baselineskip}}
%% \tikzset{level 2+/.style={level distance=5\baselineskip}}
%% \tikzset{level 3+/.style={level distance=6\baselineskip}}
%% \tikzset{level 4/.style={level distance=7\baselineskip}}
%% \tikzset{level 5+/.style={level distance=5\baselineskip}}
%% \tikzset{frontier/.style={distance from root=26\baselineskip}}
%% %% \Tree[.{\ms[np-passive-cx]{ vform & passive \\
%% %%                             subj & \sliste{ NP\ind{1} }\\[2mm]
%% %%                             comps & \sliste{ (PP[\type{by}]\ind{2}) }\\
%% %%                           } }
%% %%         \ms{ vform & psp \\
%% %%              subj & \sliste{ NP\ind{2} }\\[2mm]
%% %%              comps & \sliste{ NP\ind{1} } 
%% %%            } ]
%% \Tree[.S
%%        [.{\ibox{1} NP\ind{2}} \edge[roof]; {the boy} ]
%%        [.VP\feattab{
%%                  \subj  \sliste{ \ibox{1} NP\ind{2} },\\
%%                  \comps  \sliste{  }}
%%          [.V\feattab{
%%                  \subj  \sliste{ \ibox{1} NP\ind{2} },\\
%%                  \comps  \sliste{ \ibox{3} }} was ]
%%          [.{\ibox{3} VP\feattab{
%%                  \vform \type{passive},\\
%%                  \subj  \sliste{ \ibox{1} NP\ind{2} },\\
%%                  \comps  \sliste{ }}} \edge node[auto=left]{Passive Construction};
%%            [.V\feattab{
%%                  \vform \type{psp},\\
%%                  \subj  \sliste{ NP },\\
%%                  \comps  \sliste{ NP\ind{2} }} 
%%              [.V\feattab{
%%                  \vform \type{psp},\\
%%                  \subj  \sliste{ NP },\\
%%                  \comps  \sliste{ NP\ind{2}, \ibox{4} }} given ] \edge node[auto=left]{Schema for Passive Participles};
%%              [.{\ibox{4} NP} \edge[roof]; { the ball } ] ] ] ] ] 
%% \end{tikzpicture}
%% }
%% \caption{\label{fig-the-boy-was-given-the-ball-tseng}Analysis of \emph{The boy was given the ball} according to \citet{Tseng2007a}}
%% \end{figure}
 

\subsection{\texttt{xspace} and Abbreviations}

\subsection{\texttt{todonotes}}

\section{Software}

\begin{itemize}
\item BibDesk
\item JabRef
\end{itemize}


\section{Advanced Installation}
\label{sec-advanced-latex-installation}

If you typeset many books for \lsp, put the fonts that are contained in the directory \texttt{fonts}
into your local font directory that is used by \xelatex. Put the logos from the directory \texttt{logos} into the
search path for images. 


\subsection{Style Files and Multiple Projects}

Pathes, shell variables \ldots


\section{Checklist for Typesetters/Authors Using \latex}
\label{sec-check-typesetters}


%      <!-- Local IspellDict: en_US-w_accents -->
