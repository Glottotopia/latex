%% -*- coding:utf-8 -*-
\chapter{Guidelines for Authors}

The following sections describe the layout of various items that play a role in typesetting. 

\section{Glossed Examples}

Please gloss all examples and provide them with translations. The glossing should be done according
to the Leipzig Glossing Rules. If you need special abbreviations that are not defined by the Leipzig
Glossing Rules\footnote{
\url{http://www.eva.mpg.de/lingua/resources/glossing-rules.php}. 27.10.2013.
}
\todo{provide an example}, put the in footnote at the first occurance. If you
define several new abbreviations, put them in a table in the appendix.


\section{Figures}

\section{Tables}

\section{Crossreferences in the Text}

Please use the crossreferenceing mechanisms of your text editing/type setting software. Using such
crossreferencing mechanisms is less error-prone when you shift text blocks around and in addition
all these crossreferences will be turned into hyperlinks between document parts, which makes the
final documents much more useful.

If you have numbered example sentence, please start with (1) for every new chapter.

\section{References}


We use the \emph{Unified Style Sheet for Linguistics}, which is described here:
\url{http://celxj.org/downloads/UnifiedStyleSheet.pdf}. The \bibtex file is contained in the \latex
classes that are used for typesetting \lsp books. Please deliver a \bibtex file with all your
references together with your submissions. \bibtex can be exported from all common bibliography
tools.

The references in your \bibtex file will be typeset correctly automatically. So, provided the
\bibtex file is correct, authors do not have to worry about this. But there are some things to
observe in the main text. Please cite as shown in Table~\ref{tab-citation}.

\begin{table}[htbp]
\begin{tabular}{ll}
Author & As \citet[215]{MZ85a} has shown\\
       & As \citet[215]{MZ85a} and \citet{Bloomfield33a} have shown\\
Work   & As was shown in \citew[215]{Saussure16a}, this is a problem for \ldots\\
Work   & This is not true \citep{Saussure16a,Bloomfield33a}.
\end{tabular}
\caption{\label{tab-citation}Citation style for \lsp}
\end{table}

%% Table~\ref{tab-various-publication-types} provides examples for different publication types (book,
%% journal article, paper in an edited volume, and so on). Please refer to the bibliography at the end
%% of this book to see how the respective items are formated.

If you have an enummeration of references in the text as in \emph{As X, Y, and Z have shown}, please use
the normal punctuation of the respective language rather then special markup like `;'.

\noindent
\inlinetodostefan{Say something about decapitalization.}

\section{Checklist}

The following is a general checklist for authors. Author who use \latex should also consult the
checklist for advanced authors/typesetters in Section~\ref{sec-check-typesetters}.



















%      <!-- Local IspellDict: en_US-w_accents -->
