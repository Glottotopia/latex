%% -*- coding:utf-8 -*-
\chapter{Guidelines for Authors}

The following sections describe the layout of various items that play a role in typesetting. 

\section{Figures}

\section{Tables}

\section{Crossreferences in the Text}

Please use the crossreferenceing mechanisms of your text editing/type setting software. Using such
crossreferencing mechanisms is less error-prone when you shift text blocks around and in addition
all these crossreferences will be turned into hyperlinks between document parts, which makes the
final documents much more useful.

\section{References}


We use the \emph{Unified Style Sheet for Linguistics}, which is described here:
\url{http://celxj.org/downloads/UnifiedStyleSheet.pdf}. The \bibtex file is contained in the \latex
classes that are used for typesetting \lsp books. Please deliver a \bibtex file with all your
references together with your submissions. \bibtex can be exported from all common bibliography
tools.

The references in your \bibtex file will be typeset correctly automatically. So, provided the
\bibtex file is correct, authors do not have to worry about this. But there are some things to
observe in the main text. Please cite as shown in Table~\ref{tab-citation}.

\begin{table}[htbp]
\begin{tabular}{ll}
Author & As \citet[215]{Saussure16a} has shown\\
Work   & As was shown in \citew[215]{Saussure16a}, this is a problem for \ldots\\
\end{tabular}
\caption{\label{tab-citation}Citation style for \lsp}
\end{table}



\section{Checklist}

The following is a general checklist for authors. Author who use \latex should also consult the
checklist for advanced authors/typesetters in Section~\ref{sec-check-typesetters}.



















%      <!-- Local IspellDict: en_US-w_accents -->
